\documentclass[a4paper,12pt,titlepage,finall]{article}

\usepackage[english,russian]{babel}
\usepackage{geometry}
\usepackage{cmap}

\geometry{a4paper,left=30mm,top=30mm,bottom=30mm,right=30mm}
\setcounter{secnumdepth}{0}

\usepackage{amsmath}
\usepackage{amssymb}

\begin{document}

\begin{titlepage}
    \begin{center}
    \textsc{\small Московский государственный университет \\имени М.~В.~Ломоносова\\
    Факультет вычислительной математики и кибернетики\\Кафедра математической кибернетики\\}
    \vfill
    \textsc{\Large Отчет по преддипломной практике}\\
    ~\\
    \textbf{\large <<Программная реализация поиска минимальных обобщенных полиномов в векторном пространстве>>}\\
    ~\\
    \end{center}
    \begin{flushright}
    \vfill {Выполнил:\\
    студент 418 группы\\
    Королёв~Ф.~И.\\
    ~\\
    Научный руководитель:\\
    к.ф.-м.н. Бухман~А.~В.}
    \end{flushright}
    \begin{center}
    \vfill
    {\small Москва\\2022}
    \end{center}
\end{titlepage}

\tableofcontents
\newpage

\section{Введение}

\subsection{Основные определения}

$ E_2 $ --- множество $ \{ 0, 1 \} $.\\
Отображение $ f:\ E_2^n \rightarrow E_2 $ назовем булевой функцией от $ n $ переменных.\\
$ \mathbb{P}_2(n) $ --- множество всех булевых функций от $ n $ переменных.\\
\textit{Литерал} --- атомарная формула, кроме констант 0 и 1, или ее отрицание (например $ x $)\\
\textit{Полярность литерала} --- индикатор положительности литерала.\\
\textit{Элементарная конъюнкция} ранга $ r $ --- это выражение вида $ x_{i_1} \dots x_{i_r} $, куда литералы могут входить и с отрицательной полярностью.\\
\textit{Обобщенный полином} --- выражение вида $ \bigoplus\limits_{i = 1}^l K_i $, где $ K_i $ --- элементарные конъюнкции.\\
\textit{Длиной обобщенного полинома} назовем количество элементарных конъюнкций в нем.\\
$ \mathcal{P}(n) $ --- множество всех обобщенных полиномов от $ n $ переменных.

\subsection{Задача минимизации обобщенных полиномов}

Поскольку не существует взаимнооднозначного соответствия между $ \mathcal{P}(n) $ и $ \mathbb{P}_2(n) $, задача поиска обобщенных полиномов в коротком (в смысле длины полинома) представлении является областью активных исследований. Интерес представляют преимущества, которыми обладают обобщенные полиномы в сравнении с ДНФ. С практической точки зрения можно выделить экспериментально наблюдаемые меньшие размеры выражения \cite{exorcism4} (для метрики, отвечающей количеству конъюнктов в выражении).

Эта метрическая характеристика приводит к тому, что в производстве цифровых схем на базе логических вентилей \textsc{AND} и \textsc{XOR} уменьшены площадь схемы и время задержки \cite{delay} в сравнении со схемами на базе \textsc{AND} и \textsc{OR}. Текущее применение \textsc{AND-XOR} выражений заключается в синтезе таких двухуровневых схем, как обратимые логические\footnote{схемы, использующие вентили с характерным низким энергопотреблением, реализующующие обратимые вычисления \cite{revsynth}} \cite{reversible} и квантовые \cite{quantum}.

Рассмотрим некоторые характеристики такого представления.\\
Для $ P \in \mathcal{P}(n) $ обозначим длину полинома $ l(P) $. Введем функционал
$$ l(f) = \min\limits_{P_f \in \mathcal{P}_f} l(P_f), $$
обозначающай минимальную длину обобщенного полинома среди всех, реализующих булеву функцию $ f $. Функцией Шеннона сложности в классе обобщенных полиномов называется функция
$$ L_\text{о.п.} = \max\limits_{f \in \mathbb{P}_2(n)} l(f). $$

Существуют верхняя и нижняя оценки для функции Шеннона в классе обобщенных полиномов, представим их \cite{selezn}:
$$ \frac{2^n}{n \log_2 3} \le L_\text{о.п.}(n) \le 2 \cdot \frac{2^n}{n} (1 + \ln n) $$

Сформулируем математическую задачу: поиск минимального представления функции в классе обобщенных полиномов. Входными данными данной задачи является обобщенный полином, который представлен в виде набора входящих в него элементарных конъюнкций. Некоторые алгоритмы (например \textsc{EXMIN2} \cite{exmin2}) принимают на вход ДНФ, но на первом алгоритмическом этапе занимаются приведением его к описанному выше виду. Выходными данными служит минимальный обобщенный полином (в случае эвристического максимально сокращенный).

Некоторые исследователи предлагают алгоритмы, находящие точный полиномиальный минимум \cite{min-tau} \cite{exact6} \cite{exact}, однако до сих пор не был найден эффективный метод решения таким образом поставленной задачи для булевых функций более восьми переменных в общем случае \cite{exact8}, поэтому особый интерес представляет эвристический подход к минимизации обобщенных полиномов.

Научным сообществом было предложено множество достойно себя зарекомендовавших эвристических алгоритмов минимизации, не гарантирующих минимальности полученного в качестве результата полинома. Это свойство позволяет добиться высокого быстродействия алгоритма в сравнении с подходами к поиску точного минимума. Примерами могут служить алгоритмы Сасао \cite{exmin2}, Стергиу-Вудурис-Папаконстантину \cite{mvesopmin}, Мищенко-Перковски \cite{exorcism4}.

Помимо рассмотрения самих алгоритмов, интересны способы их тестирования и сравнения. Среди точных алгоритмов сравнивают время их работы, среди эвристических --- время и результат сокращения исходного полинома. Большинство статей \cite{exmin2} \cite{mint} \cite{exorcism4}, предлагающих эвристические подходы к решению данной задачи, содержат в себе раздел, посвященный результатам экспериментов, в котором используется много функций из бенчмарка\footnote{набор тестов} LGSynth \cite{benchmark}.

В курсовой работе за третий курс был предложен метод дискретного аналога покоординатного спуска с прибавлением нулевых обобщенных полиномов. Этот эвристический подход продолжает исследоваться в рамках данной работы.

\subsection{Метод прибавления нулевых полиномов}

Задан исходный обобщенный полином в виде $ \bigoplus\limits_i K_i $.

Генерация базиса нулевых обобщенных полиномов происходит следующим образом: представим себе элементарную конъюнкцию из $ n $ множителей. На позиции i-того литерала выставим один из трех элементов множества $ \{ 0, 1, x_i \} $. Очевидно, что конъюнкция тождественно равна 0 тогда и только тогда, когда на месте хотя бы одного литерала выставлен 0. Пусть на места всех $ n $ литералов выставлены элементы описанного множества, при этом хотя бы один из них равен $ 0 \equiv x_i \oplus \overline x_i $, тогда в силу дистрибутивности конъюнкции относительно исключающего или можно раскрыть скобки и получить выражение являющееся, в силу тождественного равенства нулю, нулевым обобщенным полиномом. Весь базис нулевых полиномов, построенный на множестве $ \{ 0, 1, x_i \} $ будет состоять из $ 3^n $ (всех вариантов расстановки элементов множества на $ n $ позиций) минус $ 2^n $ (всех вариантов расстановки элементов $ \{ 1, x_i \} $ на $ n $ позиций, поскольку выставив исключительно элементы данного подмножества на места литералов выражение не будет удовлетворять необходимому условию тождественного равенства 0). Итого, мощность базиса $ \left| B \right| = 3^n - 2^n. $

После генерации базиса нулевых обобщенных полиномов его элементы по некоторой стратегии прибавляются к исходному обобщенному полиному. В силу тождественного равенства нулю всех элементов базиса функция, задаваемая выражением до прибавления, совпадает с функцией, задаваемой результатом сложения (поскольку $ a \oplus 0 \equiv a $). В рамках отдельной итерации прибавления возможны сокращения вида $ a \oplus a \equiv 0 $, которые приведут к уменьшению длины выражения. Прибавление подтверждается в случае уменьшения длины полинома, работа алгоритма продолжается, пока удается сокращать полином.

Алгоритм возвращает сокращенный обобщенный полином в виде $ \bigoplus\limits_i K_i $.

\subsubsection{Предложенные улучшения метода}

\begin{enumerate}
    \item Построение базиса на множестве $ \{ 0, 1, x_i, \overline x_i \} $, что увеличит количество элементов до $ 4^n - 3^n $ (рассуждения из выведения оценки для базиса, построенного на множестве $ \{ 0, 1, x_i \} $, повторяются). Это негативно скажется на времени работы программы, но также может дать улучшение финальных результатов (получение более коротких полиномов).
    \item Изменение стратегии обхода элементов базиса в попытках сократить полином. Поскольку оптимизируемая функция (длина полинома) не является дифференцируемой в пространстве полиномов (иными словами о градиентном подходе речь не идет), то изменение стратегии прибавлений может привести к другому, потенциально лучшему, результату.
    \item Прибавление пар элементов базиса. Пара элементов может частично сократиться друг с другом, после чего результат может сократить исходный полином.
\end{enumerate}

\newpage

\section{Постановка задачи}

Представить примеры используемых в научной литературе методов минимизации обобщенных полиномов, а так же типичный набор тестов для их сравнения между собой.

Проанализировать предложенный метод следующими критериями:
\begin{itemize}
    \item доказать или опровергнуть существование не оптимизирующихся полиномов, которые могут быть записаны в более коротком виде
    \item проверить гипотезу улучшения результата путем расширения базиса нулевых полиномов
    \item проверить гипотезу улучшения результата путем пробы разных стратегий прибавления элементов базиса
    \item проверить гипотезу улучшения результата путем прибавления пар элементов базиса
\end{itemize}

\newpage

\section{Полученные результаты}

\subsection{Обзорная часть}

\subsubsection{Некоторые опубликованные эвристические подходы}

В 1993 году Цутому Сасао представил алгоритм, названный \textsc{EXMIN2} \cite{exmin2}.
В статье описаны 10 правил преобразования (с иллюстрациями на карте Карно), которые задействуются в дальнейшем. На вход алгоритму подается булева функция в ДНФ представлении. Идея алгоритма в следующем:
\begin{itemize}
    \item конвертировать исходную ДНФ в непересекающуюся ДНФ (disjoint sum-of-products)\footnote{может быть преобразован к обобщенным полиномам элементарной заменой операции \textsc{OR} на \textsc{XOR}} с помощью алгоритма \cite{sop2dsop}, работающего за $ O \left( 2^{n - 1} \right) $.
    \item multi-output\footnote{функции с многомерным $ E_2^m,\ m > 1 $ выходом} функции декомпозировать в набор single-output\footnote{функции с одномерным $ E_2^1 $ выходом} функций, после чего работать над минимизацией каждой из них
    \item используя правила преобразования, уменьшить полином до тех пор, пока это итеративно улучшает результат
\end{itemize}

Некоторые исследователи ссылаются на работу 1996 года Томаша Козловского с описанным алгоритмом минимизации обобщенных полиномов \textsc{MINT} \cite{mint}. К сожалению, его оригинальный труд найти не удалось. Рассмотренный далее \textsc{EXORCISM4} в среднем работает быстрее \textsc{MINT} в 25 раз (на 25 тестах \cite{exorcism4}, среди которых алгоритм Козловского вырвался вперед по быстродействию и результатам минимизации только на функции \texttt{rd84}\footnote{the input weight function with 8 inputs and 4 outputs} \cite{benchmark}).

В 1996 Дебатош Дебнатх и Цутому Сасао публикуют GRMIN2 \cite{grmin2}.
Он, по аналогии с \textsc{EXMIN2} использует правила преобразования, всего их 8. Алгоритм в первую очередь минимизирует длину полинома, а также второстепенно --- количество литералов в выражении. На вход принимает функцию в ДНФ представлении. Имеет следующие основные пункты:
\begin{itemize}
    \item из данной ДНФ генерируются DSOP (disjoint sum-of-products) \cite{exmin2} и PSDRM (pseudo Reed-Muller form) \cite{psdrm}. Опционально выбирается один из них (например полином с наименьшей длиной) или на вход алгоритму подаются оба полинома и выбирается лучший из оптимизированных
    \item далее полином проходит минимизацию в двух этапах по правилам сокращения до тех пор, пока его удается сократить
    \item затем длину полинома временно увеличивают, возвращаясь к этапам минимизации. Это необходимо, поскольку доказано, что множество правил, не имеющих ни одного <<удлиняющего>>, может не иметь траектории применения до точного минимума функции в представлении обобщенных полиномов \cite{convergence}. Шаг продолжается до тех пор, пока удается сокращать полином
    \item запускается финальный цикл, применяющий свое подмножество правил сокращения, снова до тех пор, пока сокращения удаются
\end{itemize}

В 2001 выпущена статья Алана Мищенко и Марека Перковски с уже четвертой версией алгоритма \textsc{EXORCISM4} \cite{exorcism4}. В методе \textsc{EXORCISM} описаны операции xlinks (читается crosslinks), применяемые к парам конъюнктов, основные этапы звучат таким образом:
\begin{itemize}
    \item алгоритм принимает на вход булеву функцию в виде multi-output непересекающихся (в покрытии) кубов
    \item применить всевозможные primary xlinks с приоритетом у наиболее близких пар (по расстоянию Хэмминга)
    \item применить всевозможные secondary xlinks аналогичным способом
    \item если удалось применить хотя бы один secondary xlink, то вернуться к этапу с primary links
\end{itemize}
Замечанием к алгоритму служит следующий приоритет: при любом xlinking-е при прочих равных выбирается пара, применяемая к конъюнктам, содержащим большее количество литералов (в статье это называется степенью конъюнкта). Улучшения в \textsc{EXORCISM4} коснулись вычисления начального покрытия, увеличение пространства поиска с помощью применения большего числа трансформаций. Преимуществом является быстродействие и хороший (в сравнении с другими алгоритмами \cite{exmin2} \cite{mint}) результат сокращения.

В 2008 публикуют алгоритм минимизации обобщенных полиномов min-tau2 \cite{min-tau2} Японские ученые.
Алгоритм принимает на вход полином в виде множества элементарных конъюнкций. Особенность алгоритма в том, что его наивная реализация подразумевает полный перебор комбинаций конъюнктов, что гарантирует нахождения оптимального решения, но при этом является крайне трудоемкой. Однако с помощью представленные в \cite{min-tau} теоремы позволяют значительно сократить время работы алгоритма \texttt{min-tau}, что дает возможность находить решения для функций от 6 переменных на практике. \texttt{min-tau2} же, в свою очередь ускоряет процедуру поиска, что являлось узким местом оригинального алгоритма. Метод находит точный минимум большого количества бенчмарк-функций.

\subsubsection{Часто встречаемые бенчмарки}

В подавляющем большинстве статей, посвященных эвристическим методам минимизации обобщенных полиномов, фигурируют следующие функции для сравнения алгоритмов:
\begin{itemize}
    \item single-output функции из бенчмарка LGSynth
    \item multi-output функции из бенчмарка LGSynth
    \item multi-output арифметические функции
\end{itemize}

\textbf{LGSynth.}
LGSynth --- бенчмарк, разработанный для логического синтеза и оптимизации, использовался совместно с MCNC International Workshop. Тестируемые функции хранятся в файлах разрешений \texttt{*.pla}, \texttt{*.blif}: функции в них задаются набором элементарных конъюнкций, также к ним есть документация \cite{benchmark}.

Основная масса функций из этого бенчмарка, появляющихся в статьях в качестве тестов, является multi-output. Примеры названий функций: \texttt{9sym}\footnote{its output is 1 if and only if the number of ones in the input pattern is 3, 4, 5 or 6} (single-output), \texttt{5xp1} (multi-output), \texttt{clip} (multi-output).

\textbf{Arithmetic functions.}
Арифметические функции --- multi-output функции, опишем некоторые из них.\\
\texttt{adr}$ n $ --- $ n $-битный сумматор без входа для бита переноса.\\
\texttt{inc}$ n $ инкрементирует $ n $-битное число.\\
\texttt{wgt}$ n $ считает количество возведенных в единицу битов $ n $-битного числа.

В таблице \ref{table_arithmetic} формулами описаны используемые в тестах арифметические функции $ n $ переменных ($ A,\ B \in B^n $).

\begin{table}[h!]
\centering
\begin{tabular}{ |c||c|c||c|c|c|c|c| }
\hline
\textbf{name}   & \texttt{\bf in} & \texttt{\bf out}    & \textbf{function} \\
\hline\hline
\texttt{adr} & $ 2 n $ & $ n + 1 $                      & $ A + B $ \\
\hline
\texttt{inc} & $ n $   & $ n + 1 $                      & $ A + 1 $ \\
\hline
\texttt{log} & $ n $   & $ n $                          & $ \frac{2^n - 1}{n} \times \log_2 (A + 1) $ \\
\hline
\texttt{mlp} & $ 2 n $ & $ 2 n $                        & $ A \times B $ \\
\hline
\texttt{nrm} & $ 2 n $ & $ n + 1 $                      & $ \sqrt{A^2 + B^2} + 0.5 $ \\
\hline
\texttt{rdm} & $ n $   & $ n $                          & $ (5 A + 1) \mod 2^n $ \\
\hline
\texttt{rot} & $ n $   & $ \lceil n / 2 \rceil $        & $ \lfloor \sqrt{A} + 0.5 \rfloor $ \\
\hline
\texttt{sqr} & $ n $   & $ 2 n $                        & $ A^2 $ \\
\hline
\texttt{wgt} & $ n $   & $ \lceil \log_2 n \rceil + 1 $ & $ \sum\limits_{i = 1}^n a_i $ \\
\hline
\end{tabular}
\caption{Арифметические операции}
\label{table_arithmetic}
\end{table}

\subsubsection{Некоторое обобщение}

В таблице \ref{table_benchmark} показаны некоторые результаты (количества конъюнктов по окончании работы алгоритма) минимизации функций опубликованными эвристическими алгоритмами на некоторых тестах бенчмарка LGSynth и арифметических функциях.

\begin{table}[h!]
\centering
\begin{tabular}{ |c||c|c||c|c|c|c|c| }
\hline
\textbf{benchmark} & \texttt{\bf in} & \texttt{\bf out} & \textsc{EXORCISM4} & \textsc{GRMIN2} & \textsc{MINT} & \textsc{EXMIN2} & \texttt{min-tau2} \\
\hline\hline
\texttt{9sym}   & 9  & 1  & 51 & 51 & 51 & 53 & – \\
\hline
\texttt{life}   & 9  & 1  & 48 & 49 & 51 & 54 & – \\
\hline
\texttt{ryy6}   & 16 & 1  & 40 & –  & 40 & 40 & – \\
\hline
\texttt{sym10}  & 10 & 1  & 79 & 82 & 82 & 84 & – \\
\hline\hline
\texttt{5xp1}   & 7  & 10 & 31 & 32 & 32 & 34 & – \\
\hline
\texttt{clip}   & 9  & 5  & 63 & 67 & 64 & 68 & – \\
\hline
\texttt{m181}   & 15 & 9  & 29 & 29 & 29 & 29 & – \\
\hline
\texttt{sqrt8}  & 8  & 4  & 17 & –  & –  & –  & 17 \\
\hline\hline
\texttt{adr4}   & 8  & 5  & –  & 31 & –  & –  & – \\
\hline
\texttt{log8}   & 8  & 8  & –  & 96 & –  & –  & – \\
\hline
\texttt{mlp3}   & 6  & 6  & –  & –  & –  & 18 & 18 \\
\hline
\texttt{inc8}   & 8  & 9  & –  & 15 & –  & –  & 15 \\
\hline
\end{tabular}
\caption{Результаты алгоритмов по минимизации обобщенных полиномов}
\label{table_benchmark}
\end{table}

В силу особенности алгоритма \texttt{min-tau2}, для него были выбраны специфичные бенчмарки в оригинальной работе \cite{min-tau2}, поэтому в таблице \ref{table_benchmark} стоит много прочерков для этого алгоритма. Приведем небольшую таблицу \ref{table_mintau} сравнения \texttt{min-tau2} с \textsc{EXMIN2} на некоторых арифметических тестах.

\begin{table}[h!]
\centering
\begin{tabular}{ |c||c|c||c|c|c| }
\hline
\textbf{benchmark} & \texttt{\bf in} & \texttt{\bf out} & \texttt{min-tau2} & \textsc{EXMIN2} & \texttt{mt2} time [s] \\
\hline\hline
\texttt{nrm3} & 6 & 4 & 21 & 26 & 1047388.4 \\
\hline
\texttt{rot6} & 6 & 4 & 16 & 17 & 122.1     \\
\hline
\texttt{sqr4} & 4 & 8 & 11 & 15 & 583.3     \\
\hline\hline
\texttt{log4} & 4 & 4 & 10 & 10 & 0.24      \\
\hline
\texttt{inc6} & 6 & 7 & 11 & 11 & 0.20      \\
\hline
\texttt{rdm6} & 6 & 6 & 15 & 15 & 107065.6  \\
\hline
\end{tabular}
\caption{Некоторые результаты метода минимизации \texttt{min-tau2}}
\label{table_mintau}
\end{table}

Некоторые ученые также сравнивают быстродействия своих методов с уже опубликованными, среди представленных сильно выделяются быстродействием \textsc{EXORCISM4}, а большим временем работы --- \texttt{min-tau2}.

\subsection{Минимизация полиномов методом покоординатного спуска}

\subsubsection{Существование не оптимизируемых полиномов}

В процессе тестирования программы были найдены не оптимизируемые данным методом полиномы, которые, тем не менее, удалось обнаружить в более короткой форме. Примерами таких полиномов могут стать:
\begin{enumerate}
    \item $ x_1 x_2 x_3 \overline x_4 \oplus x_1 x_2 x_3 \overline x_5 \oplus x_1 x_2 \overline x_3 x_4 \oplus x_1 \overline x_2 x_3 x_5 $
    \item $ x_4 x_5 \oplus x_1 x_2 x_3 x_4 \oplus x_1 x_2 x_4 x_5 \oplus x_1 \overline x_2 \overline x_3 x_4 \oplus x_1 x_3 x_4 x_5 \oplus \overline x_1 \overline x_2 \overline x_3 \overline x_4 \oplus x_2 x_3 x_4 x_5 \oplus \overline x_2 \overline x_3 x_4 x_5 $
    \item $ x_1 x_2 x_4 \overline x_5 \oplus \overline x_2 \overline x_3 x_4 \overline x_5 \oplus \overline x_1 x_2 x_3 x_4 x_5 \oplus \overline x_1 x_2 x_3 x_4 \overline x_6 \oplus x_2 x_3 x_4 \overline x_5 x_6 \oplus x_1 x_2 x_3 x_4 x_5 x_6 \oplus x_1 x_2 x_3 x_4 \overline x_5 \overline x_6 $
\end{enumerate}

Ни один из этих полиномов не сокращается ни оригинальным алгоритмом, ни алгоритмом с использованием любой комбинации из ниже описанных методов. Однако они представимы соответственно в более коротких формах:
\begin{enumerate}
    \item $ x_1 x_2 x_4 \oplus x_1 x_3 x_5 $
    \item $ x_1 x_2 x_4 \oplus x_1 \overline x_3 x_4 \oplus \overline x_1 x_2 x_4 x_5 \oplus \overline x_1 \overline x_2 \overline x_3 \overline x_4 \oplus \overline x_1 x_3 x_4 x_5 $
    \item $ x_1 x_2 x_4 \overline x_5 \oplus x_2 x_3 x_4 x_6 \oplus \overline x_2 \overline x_3 x_4 \overline x_5 \oplus x_2 x_3 x_4 \overline x_5 \overline x_6 $
\end{enumerate}

Также к несокращаемым данным методом отнесся один из часто встречаемых в научных исследований single-output тест из бенчмарка LGSynth: \texttt{9sym}.

\subsubsection{Расширение базиса нулевых полиномов}

Генерация базиса нулевых полиномов на множестве $ \{ 0, 1, x_i, \overline x_i \} $, как говорилось выше, приводит к увеличению его мощности с $ \left| 3^n - 2^n \right| $ до $ \left| 4^n - 3^n \right| $. Использование расширенного базиса позволило добиться лучшего результата на нескольких тестах. Входные данные сокращаемых полиномов выглядят так:
\begin{enumerate}
    \item $ x_1 \oplus x_2 \oplus x_3 \oplus \overline x_1 x_2 \oplus \overline x_1 \overline x_3 \oplus \overline x_2 \overline x_3 \oplus x_1 \overline x_2 x_3 \oplus \overline x_1 \overline x_2 \overline x_3 $
    \item $ \overline x_2 \overline x_4 \oplus x_1 \overline x_2 x_3 \oplus x_1 x_3 \overline x_4 \oplus \overline x_1 \overline x_3 \overline x_4 \oplus x_2 \overline x_3 x_4 \oplus \overline x_2 \overline x_3 \overline x_4 \oplus x_1 x_2 x_3 x_4 \oplus x_1 x_2 \overline x_3 \overline x_4 $
    \item $ \overline x_1 \overline x_3 \oplus x_2 x_3 \oplus \overline x_2 \overline x_4 \oplus x_3 x_4 \oplus x_1 x_2 x_4 \oplus x_2 \overline x_3 \overline x_4 \oplus x_1 x_2 \overline x_3 \overline x_4 \oplus x_1 \overline x_2 \overline x_3 \overline x_4 $
    \item $ x_3 \oplus x_4 \oplus x_1 \overline x_2 \oplus \overline x_1 \overline x_4 \oplus x_1 x_2 \overline x_4 \oplus x_1 x_3 \overline x_4 \oplus x_1 \overline x_3 \overline x_4 \oplus x_1 x_2 x_3 x_4 \oplus \overline x_1 \overline x_2 \overline x_3 \overline x_4 $
    \item $ \overline x_1 \overline x_6 \oplus x_2 x_4 \oplus x_3 x_5 \oplus x_3 x_6 \oplus \overline x_5 \overline x_6 \oplus x_1 \overline x_3 \overline x_6 \oplus x_1 \overline x_5 \overline x_6 \oplus \overline x_1 x_2 \overline x_3 \oplus x_2 x_3 x_4 \oplus x_2 x_3 \overline x_4 \oplus x_2 x_3 x_5 \oplus x_2 x_3 \overline x_6 \oplus x_3 x_5 \overline x_6 \oplus x_1 x_3 x_5 \overline x_6 \oplus \overline x_1 \overline x_3 \overline x_5 \overline x_6 \oplus x_2 x_3 x_4 x_5 \oplus x_2 \overline x_3 \overline x_5 \overline x_6 \oplus x_1 \overline x_2 x_3 \overline x_4 x_5 \overline x_6 $
\end{enumerate}

Результат работы алгоритма на заданных функциях с базисом мощности $ \left| 3^n - 2^n \right| $:
\begin{enumerate}
    \item $ x_3 \oplus x_1 \overline x_2 \oplus \overline x_1 \overline x_3 \oplus \overline x_2 \overline x_3 \oplus x_1 \overline x_2 x_3 \oplus \overline x_1 \overline x_2 \overline x_3 $
    \item $ \overline x_2 \overline x_4 \oplus x_1 \overline x_2 x_3 \oplus x_1 x_3 \overline x_4 \oplus \overline x_1 \overline x_3 \overline x_4 \oplus x_2 \overline x_3 x_4 \oplus \overline x_2 \overline x_3 \overline x_4 \oplus x_1 x_2 x_3 x_4 \oplus x_1 x_2 \overline x_3 \overline x_4 $
    \item $ \overline x_1 \overline x_3 \oplus x_2 x_3 \oplus \overline x_2 \overline x_4 \oplus x_3 x_4 \oplus x_1 x_2 x_4 \oplus x_2 \overline x_3 \overline x_4 \oplus x_1 x_2 \overline x_3 \overline x_4 \oplus x_1 \overline x_2 \overline x_3 \overline x_4 $
    \item $ x_3 \oplus x_4 \oplus x_1 \overline x_2 \oplus \overline x_1 \overline x_4 \oplus x_1 x_2 \overline x_4 \oplus x_1 x_3 \overline x_4 \oplus x_1 \overline x_3 \overline x_4 \oplus x_1 x_2 x_3 x_4 \oplus \overline x_1 \overline x_2 \overline x_3 \overline x_4 $
    \item $ \overline x_1 \overline x_6 \oplus x_3 x_6 \oplus \overline x_5 \overline x_6 \oplus x_1 \overline x_3 \overline x_6 \oplus x_1 \overline x_5 \overline x_6 \oplus \overline x_1 x_2 \overline x_3 \oplus x_2 x_3 \overline x_4 \oplus x_2 x_3 \overline x_6 \oplus x_2 \overline x_3 x_4 \oplus \overline x_2 x_3 x_5 \oplus x_3 x_5 \overline x_6 \oplus x_1 x_3 x_5 \overline x_6 \oplus \overline x_1 \overline x_3 \overline x_5 \overline x_6 \oplus x_2 x_3 x_4 x_5 \oplus x_2 \overline x_3 \overline x_5 \overline x_6 \oplus x_1 \overline x_2 x_3 \overline x_4 x_5 \overline x_6 $
\end{enumerate}

Результат работы алгоритма на заданных функциях с базисом мощности $ \left| 4^n - 3^n \right| $:
\begin{enumerate}
    \item $ x_3 \oplus \overline x_1 \overline x_3 $
    \item $ x_1 \overline x_2 x_3 \oplus x_1 x_3 \overline x_4 \oplus \overline x_1 \overline x_3 \overline x_4 \oplus x_2 \overline x_3 x_4 \oplus \overline x_2 x_3 \overline x_4 \oplus x_1 x_2 x_3 x_4 \oplus x_1 x_2 \overline x_3 \overline x_4 $
    \item $ \overline x_1 \overline x_3 \oplus x_2 x_3 \oplus \overline x_2 \overline x_4 \oplus x_3 x_4 \oplus x_1 x_2 x_4 \oplus x_1 \overline x_2 \overline x_3 \overline x_4 \oplus \overline x_1 x_2 \overline x_3 \overline x_4 $
    \item $ \overline x_3 \oplus x_1 \overline x_2 \oplus x_1 x_2 \overline x_4 \oplus x_1 x_2 x_3 x_4 \oplus \overline x_1 \overline x_2 \overline x_3 \overline x_4 $
    \item $ x_3 x_6 \oplus \overline x_3 \overline x_6 \oplus \overline x_1 x_2 \overline x_3 \oplus x_2 x_3 \overline x_4 \oplus x_2 x_3 \overline x_6 \oplus x_2 \overline x_3 x_4 \oplus \overline x_2 x_3 x_5 \oplus x_2 x_3 x_4 x_5 \oplus x_2 \overline x_3 \overline x_5 \overline x_6 \oplus x_1 \overline x_2 x_3 \overline x_4 x_5 \overline x_6 $
\end{enumerate}

В таблице \ref{table_result} выписаны длины результатов работы алгоритма с использованием разных базисов: построенного на множестве $ \{ 0, 1, x_i \} $, а так же на множестве $ \{ 0, 1, x_i, \overline x_i \} $. Из нее видно, что использование расширенного базиса значительно улучшает качество ответа алгоритма.

Стоит, однако, упомянуть худшую асимптотику по памяти, а так же по времени работы (стоимость одной итерации алгоритма пропорциональна мощности базиса нулевых полиномов).

\begin{table}[h!]
\centering
\begin{tabular}{ |c||c|c| }
\hline
тест & $ \left| 3^n - 2^n \right| $ & $ \left| 4^n - 3^n \right| $ \\
\hline\hline
1 & 6  & 2  \\
\hline
2 & 8  & 7  \\
\hline
3 & 8  & 7  \\
\hline
4 & 9  & 5  \\
\hline
5 & 16 & 10 \\
\hline
\end{tabular}
\caption{Сравнение длин полиномов с применением базисов разных мощностей}
\label{table_result}
\end{table}

Обратим внимание на то, что для всех тестов, кроме первого, удалось обнаружить более короткие представления в сравнении с возвращаемыми алгоритмом на расширенном базисе решениями:
\begin{enumerate}
    \item $ \overline x_1 \oplus x_1 x_3 $
    \item $ x_1 x_3 \oplus x_2 \overline x_3 \oplus \overline x_1 \overline x_2 \overline x_4 $
    \item $ x_3 \oplus x_1 x_2 x_4 \oplus \overline x_1 \overline x_3 x_4 \oplus x_2 x_3 x_4 $
    \item $ x_1 \oplus \overline x_3 \oplus x_1 x_2 \overline x_3 x_4 \oplus \overline x_1 \overline x_2 \overline x_3 \overline x_4 $
    \item $ \overline x_2 \overline x_6 \oplus \overline x_1 x_2 \overline x_3 \oplus x_2 \overline x_3 x_4 \oplus \overline x_2 x_3 \overline x_5 \oplus x_2 x_3 x_4 \overline x_5 \oplus x_2 \overline x_3 x_5 \overline x_6 \oplus x_1 \overline x_2 x_3 \overline x_4 x_5 \overline x_6 $
\end{enumerate}

\subsubsection{Выбор стратегии последовательности прибавлений}

В текущей реализации алгоритма итерации совершаются <<вслепую>>, то есть при выборе прибавляемого нулевого полинома не используется информация о текущем состоянии сокращаемого полинома. Это приводит к тому, что покоординатный спуск упирается в локальный минимум: алгоритм при неудачном прибавлении каждого полинома из базиса считает, что полином более сократить невозможно и завершает работу. Однако, был найден тест, на котором разные последовательности прибавлений дают качественно разные результаты (программа работала с расширенным базисом).
~\\
\texttt{вход}\\
$ \overline x_1 \overline x_6 \oplus x_2 x_4 \oplus x_3 x_5 \oplus x_3 x_6 \oplus \overline x_5 \overline x_6 \oplus x_1 \overline x_3 \overline x_6 \oplus x_1 \overline x_5 \overline x_6 \oplus \overline x_1 x_2 \overline x_3 \oplus x_2 x_3 x_4 \oplus x_2 x_3 \overline x_4 \oplus x_2 x_3 x_5 \oplus x_2 x_3 \overline x_6 \oplus x_3 x_5 \overline x_6 \oplus x_1 x_3 x_5 \overline x_6 \oplus \overline x_1 \overline x_3 \overline x_5 \overline x_6 \oplus x_2 x_3 x_4 x_5 \oplus x_2 \overline x_3 \overline x_5 \overline x_6 \oplus x_1 \overline x_2 x_3 \overline x_4 x_5 \overline x_6 $
~\\
последовательность результативных прибавлений с траекторией попыток прибавления, совпадающей с порядком их генерации \texttt{(1)}\\
$ \overline x_5 \overline x_6 \oplus x_1 \overline x_5 \overline x_6 \oplus \overline x_1 \overline x_5 \overline x_6 $ \\
$ \overline x_1 \overline x_6 \oplus \overline x_1 x_5 \overline x_6 \oplus \overline x_1 \overline x_5 \overline x_6 $ \\
$ x_3 x_5 \oplus x_2 x_3 x_5 \oplus \overline x_2 x_3 x_5 $ \\
$ x_2 x_4 \oplus x_2 x_3 x_4 \oplus x_2 \overline x_3 x_4 $ \\
$ x_3 x_5 \overline x_6 \oplus x_1 x_3 x_5 \overline x_6 \oplus \overline x_1 x_3 x_5 \overline x_6 $ \\
$ \overline x_1 x_5 \overline x_6 \oplus \overline x_1 x_3 x_5 \overline x_6 \oplus \overline x_1 \overline x_3 x_5 \overline x_6 $ \\
$ \overline x_1 \overline x_3 \overline x_6 \oplus \overline x_1 \overline x_3 x_5 \overline x_6 \oplus \overline x_1 \overline x_3 \overline x_5 \overline x_6 $ \\
$ \overline x_3 \overline x_6 \oplus x_1 \overline x_3 \overline x_6 \oplus \overline x_1 \overline x_3 \overline x_6 $ \\
иная последовательность результативных прибавлений \texttt{(2)}\\
$ x_3 x_5 \oplus x_2 x_3 x_5 \oplus \overline x_2 x_3 x_5 $ \\
$ \overline x_5 \overline x_6 \oplus x_1 \overline x_5 \overline x_6 \oplus \overline x_1 \overline x_5 \overline x_6 $ \\
$ x_2 x_3 \oplus x_2 x_3 x_4 \oplus x_2 x_3 \overline x_4 $ \\
$ x_3 x_5 \overline x_6 \oplus x_1 x_3 x_5 \overline x_6 \oplus \overline x_1 x_3 x_5 \overline x_6 $ \\
$ \overline x_1 \overline x_6 \oplus \overline x_1 x_5 \overline x_6 \oplus \overline x_1 \overline x_5 \overline x_6 $ \\
$ x_2 x_3 \oplus x_2 x_3 x_6 \oplus x_2 x_3 \overline x_6 $ \\
$ \overline x_1 x_5 \overline x_6 \oplus \overline x_1 x_3 x_5 \overline x_6 \oplus \overline x_1 \overline x_3 x_5 \overline x_6 $ \\
$ \overline x_1 \overline x_3 \overline x_6 \oplus \overline x_1 \overline x_3 x_5 \overline x_6 \oplus \overline x_1 \overline x_3 \overline x_5 \overline x_6 $ \\
$ x_3 x_6 \oplus x_2 x_3 x_6 \oplus \overline x_2 x_3 x_6 $ \\
$ \overline x_3 \overline x_6 \oplus x_1 \overline x_3 \overline x_6 \oplus \overline x_1 \overline x_3 \overline x_6 $ \\
результат первой последовательности (длина 10):\\
$ x_3 x_6 \oplus \overline x_3 \overline x_6 \oplus \overline x_1 x_2 \overline x_3 \oplus x_2 x_3 \overline x_4 \oplus x_2 x_3 \overline x_6 \oplus x_2 \overline x_3 x_4 \oplus \overline x_2 x_3 x_5 \oplus x_2 x_3 x_4 x_5 \oplus x_2 \overline x_3 \overline x_5 \overline x_6 \oplus x_1 \overline x_2 x_3 \overline x_4 x_5 \overline x_6 $ \\
результат второй последовательности (длина 8):\\
$ x_2 x_4 \oplus \overline x_3 \overline x_6 \oplus \overline x_1 x_2 \overline x_3 \oplus \overline x_2 x_3 x_5 \oplus \overline x_2 x_3 x_6 \oplus x_2 x_3 x_4 x_5 \oplus x_2 \overline x_3 \overline x_5 \overline x_6 \oplus x_1 \overline x_2 x_3 \overline x_4 x_5 \overline x_6 $ \\
предложенное представление (длина 7):\\
$ \overline x_2 \overline x_6 \oplus \overline x_1 x_2 \overline x_3 \oplus x_2 \overline x_3 x_4 \oplus \overline x_2 x_3 \overline x_5 \oplus x_2 x_3 x_4 \overline x_5 \oplus x_2 \overline x_3 x_5 \overline x_6 \oplus x_1 \overline x_2 x_3 \overline x_4 x_5 \overline x_6 $

Из этого примера следует, что качество работы алгоритма зависит от стратегии обхода нулевых полиномов в попытке сократить ими исходный.

\subsubsection{Прибавления пар элементов}

Как уже упоминалось, в работе Цутому Сасао \cite{convergence} было представлено доказательство того, что набор исключительно уменьшающих правил может не приводить к точному минимуму при любом выборе последовательности их применения. Это наталкивает на мысль, что покоординатный спуск, принимающий итерацию только в случае удачного сокращения длины полинома также будет обладать этим нежелательным свойством.

Идея прибавлять пары нулевых полиномов вместо последовательного прибавления элементов базиса призвана временно (на глубину 1) принять увеличивающее правило, чтобы снабдить алгоритм необходимым условием сходимости набора правил из \cite{convergence}.

Однако множественные попытки улучшить результат, используя эту модификацию алгоритма, в том числе в разных комбинационных вариациях с описанными в данном подразделе, привели исключительно к значительному увеличению времени работы программы. Результат работы алгоритма не удалось улучшить ни на одном из проверяемых тестах.

\subsection{Полученные результаты}

Были озвучены подходы к оцениванию качества работы алгоритма, решающего задачу минимизации обобщенных полиномов на примере нескольких предложенных научным сообществом эвристических методов.

Была написана программа для нахождения потенциальных возможностей развития подхода к эвристическому сжатию обобщенного полинома методом прибавления нулевых полиномов. Оправдали свое рассмотрение с точки зрения улучшения результата работы алгоритма следующие модификации:
\begin{itemize}
    \item расширение базиса нулевых полиномов путем добавления дополнительного потенциального множителя в процессе генерации
    \item изменение стратегии обхода нулевых полиномов в попытке сократить им исходный путем прибавления
\end{itemize}

Не принес практической пользы подход с прибавлением пар нулевых полиномов. Однако, в соответствии с \cite{convergence}, необходимым условием сходимости эвристического алгоритма к точному минимуму является наличие <<раздувающего>> полином правила. Использование пар может оказаться эффективнее с точки зрения результата работы алгоритма, если детерминировать порядок их выбора.

\newpage

\section{План дальнейших работ}

-

\newpage

\begin{raggedright}
\addcontentsline{toc}{section}{Литература}
\begin{thebibliography}{99}
    \bibitem{exorcism4} Mishchenko~A., Perkowski~M. Fast heuristic minimization of exclusive-sums-of-products. – 2001.
    \bibitem{delay} Kalay~U., Hall~D.~V., Perkowski~M.~A. A minimal universal test set for self-test of EXOR-sum-of-products circuits //IEEE Transactions on Computers. – 2000. – Т.~49. – №.~3. – С.~267-276.
    \bibitem{reversible} Yang~G. et al. Majority-based reversible logic gates //Theoretical computer science. – 2005. – Т.~334. – №.~1-3. – С.~259-274.
    \bibitem{quantum} Iwama~K., Kambayashi~Y., Yamashita~S. Transformation rules for designing CNOT-based quantum circuits //Proceedings of the 39th annual Design Automation Conference. – 2002. – С.~419-424.
    \bibitem{revsynth} Shende~V.~V. et al. Reversible logic circuit synthesis //Proceedings of the 2002 IEEE/ACM international conference on Computer-aided design. – 2002. – С.~353-360.
    \bibitem{selezn} Селезнева~С.~Н., Шуплецов~М.~С., Дайняк~А.~Б. Булевы функции и полиномы //Москва – 2006. – С.~9-13.
    \bibitem{exmin2} Sasao~T. EXMIN2: A simplification algorithm for exclusive-OR-sum-of-products expressions for multiple-valued-input two-valued-output functions //IEEE Transactions on Computer-Aided Design of Integrated Circuits and Systems. – 1993. – Т.~12. – №.~5. – С.~621-632.
    \bibitem{min-tau} Hirayama~T., Nishitani~Y., Sato~T. A faster algorithm of minimizing AND-EXOR expressions //IEICE transactions on fundamentals of electronics, communications and computer sciences. – 2002. – Т.~85. – №.~12. – С.~2708-2714.
    \bibitem{exact6} Gaidukov~A. Algorihm to derive minimum ESOP for 6-variable function //5th International Workshop on Boolean Problems, Sept.~2002. – 2002.
    \bibitem{exact} Sasao~T. An exact minimization of AND-EXOR expressions using reduced covering functions //Proc. of the Synthesis and Simulation Meeting and International Interchange. – 1993. – С.~374-383.
    \bibitem{exact8} Stergiou~S., Papakonstantinou~G. Exact minimization of ESOP expressions with less than eight product terms //Journal of Circuits, Systems, and Computers. – 2004. – Т.~13. – №.~01. – С.~1-15.
    \bibitem{mvesopmin} Stergiou~S., Voudouris~D., Papakonstantinou G. Multiple-value exclusive-or sum-of-products minimization algorithms //IEICE TRANSACTIONS on Fundamentals of Electronics, Communications and Computer Sciences. – 2004. – Т.~87. – №.~5. – С.~1226-1234.
    \bibitem{mint} Kozlowski~T. Application of exclusive-OR logic in technology independent logic optimization //PhD Thesis, Bristol University. – 1996.
    \bibitem{benchmark} Yang~S. Logic synthesis and optimization benchmarks user guide: version 3.0. – Research Triangle Park, NC, USA : Microelectronics Center of North Carolina (MCNC), 1991. – С.~502-508.
    \bibitem{sop2dsop} Sasao~T. An algorithm to derive the complement of a binary function with multiple-valued inputs //IEEE transactions on computers. – 1985. – Т.~34. – №.~02. – С.~131-140.
    \bibitem{grmin2} Debnath~D., Sasao~T. GRMIN2: A heuristic simplification algorithm for generalised Reed-Muller expressions //IEE Proceedings-Computers and Digital Techniques. – 1996. – Т.~143. – №.~6. – С.~376-384.
    \bibitem{psdrm} Sasao~T. AND-EXOR expressions and their optimization //Logic synthesis and optimization. – Springer, Boston, MA, 1993. – С.~287-312.
    \bibitem{convergence} Brand~D., Sasao~T. Minimization of AND-EXOR expressions using rewrite rules //IEEE Transactions on Computers. – 1993. – Т.~42. – №.~5. – С.~568-576.
    \bibitem{min-tau2} Hirayama~T., Nishitani~Y. Exact minimization of and-exor expressions of practical benchmark functions //Journal of Circuits, Systems, and Computers. – 2009. – Т.~18. – №.~03. – С.~465-486.
\end{thebibliography}
\end{raggedright}

\end{document}
