\documentclass[a4paper,12pt,titlepage,finall]{article}

\usepackage[english,russian]{babel}
\usepackage{geometry}
\usepackage{cmap}
\usepackage{amsmath}

\geometry{a4paper,left=30mm,top=30mm,bottom=30mm,right=30mm}

\begin{document}

\begin{titlepage}
    \begin{center}
    {\small \sc Московский государственный университет \\имени М.~В.~Ломоносова\\
    Факультет вычислительной математики и кибернетики\\Кафедра математической кибернетики\\}
    \vfill
    {\Large \sc Отчет по преддипломной практике}\\
    ~\\
    {\large \bf <<Программная реализация поиска минимальных обобщенных полиномов в векторном пространстве>>}\\
    ~\\
    \end{center}
    \begin{flushright}
    \vfill {Выполнил:\\
    студент 418 группы\\
    Королёв~Ф.~И.\\
    ~\\
    Научный руководитель:\\
    к.ф.-м.н. Бухман~А.~В.}
    \end{flushright}
    \begin{center}
    \vfill
    {\small Москва\\2022}
    \end{center}
\end{titlepage}

\tableofcontents
\newpage

% Объем отчета --- не менее 7 страниц, включая титульную и содержание. Отчет должен содержать разделы этого шаблона, но может содержать и другие разделы.

\section{Введение}

% Содержание этого раздела определяет научный руководитель. Как правило, это обоснование актуальности задачи выпускной работы, соответствие тем практики и выпускной работы и обзор известных результатов, нерешенных проблем и/или близких задач по тематике работы.

\subsubsection*{Минимизация обобщенных полиномов}

Задача минимизации обобщенных полиномов является активной областью исследований благодаря преимуществам, которыми обладают обобщенные полиномы в сравнении с ДНФ. К ним относятся:
\begin{itemize}
    \item отличные свойства тестируемости~\cite{exorcism4}
    \item экспериментально наблюдаемые меньшие размеры выражения
    \item асимптотически лучшая верхняя оценка на количество конъюнктов ($ 29 \cdot 2^{n - 7}, \ n > 6 $~\cite{bound} против $ 2^{n - 1} $ для ДНФ)
\end{itemize}

В задаче нахождения точного минимума обобщенных полиномов до сих пор не найдено эффективного решения для булевых функций более шести переменных в общем случае~\cite{exact}, поэтому особый интерес представляет эвристический подход к минимизации обобщенных полиномов.

Также, помимо рассмотрения самих алгоритмов, интересны способы их тестирования и сравнения. Большинство статей~\cite{exmin2}~\cite{mint}~\cite{exorcism4}, предлагающих эвристические подходы к решению данной задачи, содержат в себе раздел, посвященный результатам экспериментов, в котором используется много функций из бенчмарка LGSynth~\cite{benchmark}.

В рамках данной работы продолжает исследоваться дискретный аналог покоординатного спуска с прибавлением нулевых обобщенных полиномов к исходному.

\subsubsection*{Метод прибавления нулевых полиномов}

Задан исходный обобщенный полином в виде $ \bigoplus\limits_i K_i $, где $ K_i $ --- элементарные конъюнкции.

Генерация базиса нулевых обобщенных полиномов происходит следующим образом: представим себе элементарную конъюнкцию из $ n $ множителей. На позиции i-того литерала выставим один из трех элементов множества $ \{ 0, 1, x_i \} $. Очевидно, что конъюнкция тождественно равна 0 тогда и только тогда, когда на месте хотя бы одного литерала выставлен 0. Пусть на места всех $ n $ литералов выставлены элементы описанного множества, при этом хотя бы один из них равен 0, тогда в силу дистрибутивности конъюнкции относительно исключающего или можно раскрыть скобки и получить выражение являющееся, в силу тождественного равенства нулю, нулевым обобщенным полиномом. Весь базис нулевых полиномов, построенный на множестве $ \{ 0, 1, x_i \} $ будет состоять из $ 3^n $ (всех вариантов расстановки элементов множества на $ n $ позиций) минус $ 2^n $ (всех вариантов расстановки элементов $ \{ 1, x_i \} $ на $ n $ позиций, поскольку выставив исключительно элементы данного подмножества на места литералов выражение не будет удовлетворять необходимому условию тождественного равенства 0). Итого, мощность базиса $ \left| B \right| = 3^n - 2^n. $

После генерации базиса нулевых обобщенных полиномов его элементы по некоторой стратегии прибавляются к исходному обобщенному полиному. В силу тождественного равенства нулю всех элементов базиса функция, задаваемая выражением до прибавления, совпадает с функцией, задаваемой результатом сложения (поскольку $ a \oplus 0 \equiv a $). В рамках отдельной итерации прибавления возможны сокращения вида $ a \oplus a \equiv 0 $, которые приведут к уменьшению длины (количества конъюнктов в выражении) выражения.

Алгоритм возвращает сокращенный обобщенный полином в виде $ \bigoplus\limits_i K_i $.

\subsubsection*{Возможные улучшения метода}

\begin{enumerate}
    \item Построение базиса на множестве $ \{ 0, 1, x_i, \overline{x_i} \} $, что увеличит количество элементов до $ 4^n - 3^n $ (рассуждения из выведения оценки для базиса, построенного на множестве $ \{ 0, 1, x_i \} $, повторяются). Это негативно скажется на времени работы программы, но также может дать улучшение финальных результатов (получение более коротких полиномов).
    \item Изменение стратегии обхода элементов базиса в попытках сократить полином. Может привести к другому, потенциально лучшему, результату.
    \item Прибавление пар элементов базиса. Пара элементов может частично сократиться друг с другом, после чего результат может сократить исходный полином.
\end{enumerate}

\section{Постановка задачи}

% В этом разделе описывается задача, которую требовалось решить в рамках практики.

Представить примеры используемых в научной литературе методов минимизации обобщенных полиномов, а так же типичный набор тестов для их сравнения между собой.

Проанализировать предложенный метод следующими критериями:
\begin{itemize}
    \item существование не оптимизирующихся полиномов, которые могут быть записаны в более коротком виде
    \item возможность улучшить результат путем прибавления пар элементов базиса
    \item возможность улучшить результат путем расширения базиса нулевых полиномов
    \item возможность улучшить результат путем пробы разных стратегий прибавления элементов базиса
\end{itemize}

\section{Полученные результаты}

% В этом разделе подробно описываются результаты, полученные в рамках практики.

\subsection*{Обзор}

\subsubsection*{Некоторые опубликованные эвристические подходы}

В 1993 году Цутому Сасао представил алгоритм, названный {\sc EXMIN2}~\cite{exmin2}.

В статье подробно рассказано про 8 правил преобразования (с иллюстрациями на карте Карно), которые задействуются в алгоритме. Сам алгоритм
~\\
1996 MINT~\cite{mint}\\
1996 GRMIN2~\cite{grmin2}\\
2001 EXORCISM4~\cite{exorcism4}\\
2008 min-tau2~\cite{min-tau2}

\subsubsection*{Часто встречаемые бенчмарки}

В подавляющем большинстве статей, посвященных эвристическим методам минимизации обобщенных полиномов, фигурируют следующие функции для сравнения алгоритмов:
\begin{itemize}
    \item single-output функции из бенчмарка LGSynth
    \item multi-output функции из бенчмарка LGSynth
    \item multi-output арифметические функции
\end{itemize}

{\bf LGSynth.}
LGSynth --- бенчмарк, разработанный для логического синтеза и оптимизации, использовался совместно с MCNC International Workshop. Тестируемые функции хранятся в файлах разрешений \texttt{*.pla}, \texttt{*.blif}, также к ним есть документация~\cite{benchmark}.

Основная масса функций из этого бенчмарка, появляющихся в статьях в качестве тестов, является multi-output. Примеры названий функций: \texttt{9sym} (single-output), \texttt{5xp1} (multi-output), \texttt{clip}.

{\bf Arithmetic functions.} Арифметические функции --- multi-output функции, опишем некоторые из них.\\
\texttt{adr}$ n $ --- $ n $-битный сумматор без входа для бита переноса.\\
\texttt{inc}$ n $ инкрементирует $ n $-битное число.\\
\texttt{wgt}$ n $ считает количество возведенных в единицу битов $ n $-битного числа.

Ниже представлена таблица, формулами описывающее поведение используемых в тестах арифметических функций $ n $ переменных ($ A,\ B \in B^n $).

\begin{table}[h]
\centering
\begin{tabular}{ |c||c|c||c|c|c|c|c| }
\hline
{\bf name}   & \texttt{\bf in} & \texttt{\bf out}       & {\bf function} \\
\hline\hline
\texttt{adr} & $ 2 n $ & $ n + 1 $                      & $ A + B $ \\
\hline
\texttt{inc} & $ n $   & $ n + 1 $                      & $ A + 1 $ \\
\hline
\texttt{log} & $ n $   & $ n $                          & $ \frac{2^n - 1}{n} \times \log_2 (A + 1) $ \\
\hline
\texttt{mlp} & $ 2 n $ & $ 2 n $                        & $ A \times B $ \\
\hline
\texttt{nrm} & $ 2 n $ & $ n + 1 $                      & $ \sqrt{A^2 + B^2} + 0.5 $ \\
\hline
\texttt{rdm} & $ n $   & $ n $                          & $ (5 A + 1) \mod 2^n $ \\
\hline
\texttt{rot} & $ n $   & $ \lceil n / 2 \rceil $        & $ \lfloor \sqrt{A} + 0.5 \rfloor $ \\
\hline
\texttt{sqr} & $ n $   & $ 2 n $                        & $ A^2 $ \\
\hline
\texttt{wgt} & $ n $   & $ \lceil \log_2 n \rceil + 1 $ & $ \sum\limits_{i = 1}^n a_i $ \\
\hline
\end{tabular}
\caption{Арифметические операции}
\label{table_arithmetic}
\end{table}

Представим таблицу с результатами (указаны количества конъюнктов по окончании работы) минимизации некоторых функций опубликованными эвристическими алгоритмами.

\begin{table}[h]
\centering
\begin{tabular}{ |c||c|c||c|c|c|c|c| }
\hline
{\bf benchmark} & \texttt{\bf in} & \texttt{\bf out} & {\sc EXORCISM4} & {\sc GRMIN2} & {\sc MINT} & {\sc EXMIN2} & \texttt{min-tau2} \\
\hline\hline
\texttt{9sym}   & 9  & 1  & 51 & 51 & 51 & 53 & – \\
\hline
\texttt{life}   & 9  & 1  & 48 & 49 & 51 & 54 & – \\
\hline
\texttt{ryy6}   & 16 & 1  & 40 & –  & 40 & 40 & – \\
\hline
\texttt{sym10}  & 10 & 1  & 79 & 82 & 82 & 84 & – \\
\hline\hline
\texttt{5xp1}   & 7  & 10 & 31 & 32 & 32 & 34 & – \\
\hline
\texttt{clip}   & 9  & 5  & 63 & 67 & 64 & 68 & – \\
\hline
\texttt{m181}   & 15 & 9  & 29 & 29 & 29 & 29 & – \\
\hline\hline
\texttt{adr4}   & 8  & 5  & –  & 31 & –  & –  & – \\
\hline
\texttt{log8}   & 8  & 8  & –  & 96 & –  & –  & – \\
\hline
\texttt{mlp3}   & 6  & 6  & –  & –  & –  & 18 & 18 \\
\hline
\texttt{sqrt8}  & 8  & 4  & 17 & –  & –  & –  & 17 \\
\hline
\texttt{inc8}   & 8  & 9  & –  & 15 & –  & –  & 15 \\
\hline
\end{tabular}
\caption{Результаты некоторых алгоритмов по минимизации}
\label{table_benchmark}
\end{table}


\section{План дальнейших работ}

% В этом разделе приводится развернутый план работы в следующем семестре (в рамках ВКР).

-

\begin{raggedright}
\addcontentsline{toc}{section}{Литература}
\begin{thebibliography}{99}
    \bibitem{exorcism4} Mishchenko~A., Perkowski~M. Fast heuristic minimization of exclusive-sums-of-products. – 2001.
    \bibitem{bound} Gaidukov~A. Algorihm to derive minimum ESOP for 6-variable function //5th International Workshop on Boolean Problems, Sept.~2002. – 2002.
    \bibitem{exact} Sasao~T. An exact minimization of AND-EXOR expressions using reduced covering functions //Proc. of the Synthesis and Simulation Meeting and International Interchange. – 1993. – С.~374-383.
    \bibitem{exmin2} Sasao~T. EXMIN2: A simplification algorithm for exclusive-OR-sum-of-products expressions for multiple-valued-input two-valued-output functions //IEEE Transactions on Computer-Aided Design of Integrated Circuits and Systems. – 1993. – Т.~12. – №.~5. – С.~621-632.
    \bibitem{mint} Kozlowski~T. Application of exclusive-OR logic in technology independent logic optimization //PhD Thesis, Bristol University. – 1996.
    \bibitem{benchmark} Yang~S. Logic synthesis and optimization benchmarks user guide: version 3.0. – Research Triangle Park, NC, USA : Microelectronics Center of North Carolina (MCNC), 1991. – С.~502-508.
    \bibitem{grmin2} Debnath~D., Sasao~T. GRMIN2: A heuristic simplification algorithm for generalised Reed-Muller expressions //IEE Proceedings-Computers and Digital Techniques. – 1996. – Т.~143. – №.~6. – С.~376-384.
    \bibitem{min-tau2} Hirayama~T., Nishitani~Y. Exact minimization of and–exor expressions of practical benchmark functions //Journal of Circuits, Systems, and Computers. – 2009. – Т.~18. – №.~03. – С.~465-486.
    \bibitem{arithmetic} Sasao~T. Multiple-valued logic and optimization of programmable logic arrays //Computer. – 1988. – Т.~21. – №.~4. – С.~71-80.
\end{thebibliography}
\end{raggedright}

\end{document}
